%formato de plantilla que se va a utilizar
\documentclass[a4paper]{article} 
%para idioma español
\usepackage[T1]{fontenc}
\usepackage[utf8]{inputenc}
\usepackage[spanish, mexico]{babel}
\usepackage[style=mexican]{csquotes}
%gestor de espacio
\usepackage[margin=2cm,top=2cm,includefoot]{geometry}
%gestor de imagenes
\usepackage{graphicx}
%deteccion de color
\usepackage[table,xcdraw]{xcolor}
%insericion de cuadros
\usepackage[most]{tcolorbox}
%definir el estilo de lapagina
\usepackage{fancyhdr}
%gestion de hypervinculos
\usepackage[hidelinks]{hyperref}
%marca de agua
\usepackage{background}
%dibujar figuras
\usepackage{circuitikz, tikz, pgfplots}


%cabecera
\setlength{\headheight}{40pt}
\pagestyle{fancy}
\fancyhf{}
\rhead{\includegraphics[width=1.2cm]{\logo}}
\lhead{\bfseries\textcolor{letraCabecera}{Ingeniería en Telecomunicaciones, Sistemas y Electrónica}}
%\rhead{\tiny\bfseries\textcolor{letraCabecera}{Ingeniería en Telecomunicaciones, Sistemas y Electrónica}}
\renewcommand{\headrulewidth}{3pt}
\renewcommand{\headrule}{\hbox to\headwidth{\color{lineCabecera}\leaders\hrule height \headrulewidth\hfill}}

%variables de color
\definecolor{greenPortada}{HTML}{69A84F}
\definecolor{lineCabecera}{HTML}{5DADE2}
\definecolor{letraCabecera}{HTML}{399CDE} 


%variables globales
\newcommand{\logo}{img/otro.png}

%config de la marca de agua
\backgroundsetup{
    scale=1,
    color=black,
    opacity=0.05,
    angle=0,
    position=current page.center,
    vshift=-0.5cm,
    contents={\includegraphics[width=\textwidth,height=\textheight,keepaspectratio]{\logo}}
}

%comienzo del documento
\begin{document}
    %creacion de portada
    \begin{titlepage}
        \centering
        \scalebox{0.9}{\Huge\textbf{Universidad Nacional Autónoma de México}}
        \par\vspace{0.5cm}
        \scalebox{0.9}{\Huge{Facultad de Estudios Superiores Cuautitlán}}

        \begin{tikzpicture}
            \draw(0,0)--(16,0);
        \end{tikzpicture}
        \par\vspace{1cm}

        \raggedright{\huge\textbf{Laboratorio:}}
        \par\vspace{0.5cm}
        \centering
        {\huge{Sistemas basados en redes neuronales}}
        \par\vspace{0.5cm}

        \raggedright{\huge\textbf{Grupo:}}
        \par\vspace{0.5cm}
        \centering
        {\huge{1909-C}}
        \par\vspace{0.5cm}

        \raggedright{\huge\textbf{Profesor:}}
        \par\vspace{0.5cm}
        \centering
        {\huge{Ing. Garcia Garcia Maribel}}
        \par\vspace{0.5cm}

        \raggedright{\huge\textbf{Alumno:}}
        \par\vspace{0.5cm}
        \centering
        {\huge{Palomino Alfonso Edgar}}
        \par\vspace{0.5cm}

        \raggedright{\huge\textbf{Nombre de la práctica:}}
        \par\vspace{0.5cm}
        \centering
        {\huge{Perceptron}}
        \par\vspace{0.5cm}

        \raggedright{\huge\textbf{Número de la práctica:}}
        \par\vspace{0.5cm}
        \centering
        {\huge{02}}
        \par\vspace{0.5cm}

        \raggedright{\huge\textbf{Realizado:}}
        \par\vspace{0.5cm}
        \centering
        {\huge{08/09/2023}}
        \par\vspace{0.5cm}

        \raggedright{\huge\textbf{Entregado:}}
        \par\vspace{0.5cm}
        \centering
        {\huge{22/09/2023}}
        \par\vspace{0.5cm}

        \raggedright{\huge\textbf{Semestre:}}
        \par\vspace{0.5cm}
        \centering
        {\huge{2024-1}}
        \par\vspace{0.5cm}

        \vfill

    \end{titlepage}
    \clearpage

    %creacion de indice
    \tableofcontents
    \cfoot{\thepage}
    \clearpage
    %desarrollo
    \section{Antecedentes}
    \section{Objeticos}
    
\end{document}